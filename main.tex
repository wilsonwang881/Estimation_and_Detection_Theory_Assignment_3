\documentclass[11pt,letterpaper,titlepage]{article}

\usepackage{geometry}
\geometry{left=1.5cm,right=1.5cm,top=1.5cm,bottom=2.5cm}

\usepackage{setspace}
\onehalfspacing

\usepackage{bbm}

\usepackage{amsmath}

\usepackage{amssymb}

\usepackage{booktabs}

\usepackage{pifont}

\usepackage{fancyhdr}

\pagestyle{fancy}
\lhead{}
\rhead{}
\lfoot{ECEN 662 Estimation and Detection Theory}
\cfoot{\thepage}
\rfoot{Assignment 3 @Lei Wang}
\renewcommand{\headrulewidth}{0pt}
\renewcommand{\headwidth}{\textwidth}
\renewcommand{\footrulewidth}{0.4pt}
\newcommand{\RomanNumeralCaps}[1]
    {\MakeUppercase{\romannumeral #1}}

\begin{document}

\begin{enumerate}
    
    \item % Q1
    
    Want the following:
    
    \begin{equation*}
        \begin{aligned}
            \frac{p(x | \theta)}{q(T(x) | \theta)} \\
        \end{aligned}
    \end{equation*}
    
    to be a constant. We have:
    
    \begin{equation*}
        \begin{aligned}
            p(x | \theta) &= \frac{1}{\sqrt{2 \pi \sigma^2}} e^{- \frac{1}{2 \sigma^2} x^2} \\
            q(T(x) | \theta) &= q(|x| | \theta) \\
            &= f(-x \leq X \leq x | \theta) \\
            &= 2 \cdot \frac{1}{\sqrt{2 \pi \sigma^2}} e^{- \frac{1}{2 \sigma^2} x^2}
        \end{aligned}
    \end{equation*}
    
    Hence:
    
    \begin{equation*}
        \frac{p(x | \theta)}{q(T(x) | \theta)} = \frac{1}{2}
    \end{equation*}
    
    $|x|$ is a sufficient statistic.
    
    \item % Q2
    
    Due to independence and the domain each PDF is defined upon,the joint PDF is:
    
    \begin{equation*}
        \begin{aligned}
            \prod_{i=1}^{n} e^{i \theta - x_i} \mathbbm{1}_{\{i \theta, \infty\}}(x_i) &= e^{\sum_{i = 1}^{n} (i \theta - x_i)}\mathbbm{1}_{\{i \theta, \infty\}}(x_i) \\
            &= e^{\sum_{i = 1}^{n} (i \theta - x_i)} \mathbbm{1}_{\{ \theta, \infty\}}(\frac{x_i}{i}) \\
            &= e^{\sum_{i = 1}^{n}i \theta} \mathbbm{1}_{\{ \theta, \infty\}}(\frac{x_i}{i}) e^{- \sum_{i = 1}^{n} x_i} 
        \end{aligned}
    \end{equation*}
    
    Using the factorization theorem:
    
    \begin{equation*}
        \begin{aligned}
            g(T(X)|\theta) &= e^{\sum_{i = 1}^{n}i \theta} \mathbbm{1}_{\{ \theta, \infty\}}(\frac{x_i}{i}) \\
            h(X) &= e^{- \sum_{i = 1}^{n} x_i}
        \end{aligned}
    \end{equation*}
    
    Therefore $T$ is a sufficient statistic.
    
    \item % Q3
    
    The joint PDF is:
    
    \begin{equation*}
        \begin{aligned}
            \prod_{i = 1}^{n} \frac{1}{\sigma} e^{- \frac{x_i - \mu}{\sigma}} \mathbbm{1}_{\{\mu, \infty\}}(x_i) &= \prod_{i = 1}^{n} \frac{1}{\sigma} e^{- \frac{x_i - \mu}{\sigma}} \mathbbm{1}_{\{-\infty, X_{min}\}}(\mu) \\
            &= \frac{1}{\sigma^n} e^{-\frac{\sum_{i = 1}^{n} (x_i - \mu)}{\sigma}} \mathbbm{1}_{\{-\infty, X_{min}\}}(\mu) \\
            &= \frac{1}{\sigma^n} e^{\frac{n \mu}{\sigma}} e^{-\frac{\sum_{i = 1}^{n} x_i}{\sigma}} \mathbbm{1}_{\{-\infty, X_{min}\}}(\mu)
        \end{aligned}
    \end{equation*}
    
    Using the factorization theorem, one possible choice for $h(X)$ is:
    
    \begin{equation*}
        h(X) = 1
    \end{equation*}
    
    Possible sufficient statistic will be:
    
    \begin{equation*}
        \begin{aligned}
            T_1(X) &= X_{min} \\
            T_2(X) &= \sum_{i = 1}^{n} x_i
        \end{aligned}
    \end{equation*}
    
    \item % Q4
    
    Due to independence, the joint PDF is:
    
    \begin{equation*}
        \begin{aligned}
            \prod_{i = 1}^{n} \frac{1}{2i\theta} \mathbbm{1}_{\{-i(\theta-1), i(\theta+1)\}}(x_i) &= \prod_{i = 1}^{n} \frac{1}{2i\theta} \mathbbm{1}_{\{1-min(\frac{x_i}{i}), \infty\}}(\theta) \mathbbm{1}_{\{min(\frac{x_i}{i})-1, \infty\}}(\theta) \\
            &= \frac{1}{2^n n! \theta^n} \mathbbm{1}_{\{1-min(\frac{x_i}{i}), \infty\}}(\theta) \mathbbm{1}_{\{min(\frac{x_i}{i})-1, \infty\}}(\theta)
        \end{aligned}
    \end{equation*}
    
    Sufficient statistics are:
    
    \begin{equation*}
        \begin{aligned}
            T_1(X) &= 1 - min(\frac{x_i}{i}) \\
            T_2(X) &= min(\frac{x_i}{i}) - 1
        \end{aligned}
    \end{equation*}
    
    \item % Q5
    
    \item % Q6
\end{enumerate}



\end{document}
